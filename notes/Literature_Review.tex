%
\documentclass[12pt]{article}
%\usepackage[numbers]{natbib}
\usepackage{cite}

\title{Literature Review for SMORES Verification}
\author{Tarik Tosun}
\date{\today}
\usepackage[normalem]{ulem}

\begin{document}

\maketitle

In this paper, we present a system that allows users to construct modular robot behaviors by composing existing
ones.  The system verifies new behaviors while they are being developed, providing
warnings that allowing the user to quickly identify common problems before performing an expensive dynamic simulation. The problems the system can identify are self-collision, loss of quasi-static stability, and the presence of unexpected behaviors.

\section*{Composition and Verification in Modular Robotics}
\subsubsection*{Distributed Watchpoints: Debugging Large Multi-Robot Systems \cite{de2008distributed} }
Presents a message-passing scheme that implements watchpoints (conditional breakpoints)
on a distributed system of modular robots. Watchpoints are specified in a description
language based on LTL. Focuses on detecting errors during execution of a program
on a distributed system.  
\subsubsection*{A Self-Reconfigurable Modular Robot: Reconfiguration Planning and Experiments \cite{yoshida2002self} }
Presents a layered motion planner for modular robot cluster flow (``cluster flow'' is a mode
of locomotion). Cube-shaped metamodules move from the rear of a cluster to the front. At the lowest layer, individual module movements are verified by checking for self-collision and cluster connectivity.

\subsubsection*{Reconfiguration Planning for Modular Self-reconfigurable Robots \cite{casal2002}}
This is the Ph.D thesis of Arancha Casal, who Mark worked with at PARC.  Unfortunately I have not
been able to find a copy of it yet, but Mark told me it covers self-collision and
quasi-static stability detection for modular robots, so it is very relevant to our
paper.  I have emailed her old advisor (Jean-Claude Latombe, who also advised Mark)
asking for a copy.
 
\section*{Composition and Verification in Robotics }
\subsection*{PPR}
The PPR (Printable Programmable Robots) project   is developing manufacturing, design, and programming techniques to allow
novice users to easily create robots.  Design techniques focus on composing new robots from modular elements in a library. Verification
is not stressed. 

\subsubsection*{A Design Environment for the Rapid Specification and Fabrication
of Printable Robots \cite{mehta2014design} }

Presents a design environment to allow casual users to quickly and easily create
custom robots.  New robots are built by composing mechanical, electrical, and software
subsystems.  Supporting software is automatically co-generated when the user composes
functional elements (such as legs and a body).

Verification of new designs and behaviors is not stressed.  Elements are connected
at prescribed parametric interfaces. Composition focuses on structure (e.g. combine motors, propellers, and microprocessors to make
a quadrotor) rather than task-facing functionality (e.g. build me a robot that maintains
quasi-static stability while walking, and can get up onto a table).   

While the system has modular components, the robots are not ``modular robots'' in
our sense of the word.  Robots in this paper are manufactured  by cutting and folding
plastic and paper sheet material. However, the design environment is not specifically
limited to these design techniques.


\subsubsection*{Demo abstract: ROSLab - A modular programming environment for robotic
applications \cite{bezzo2014demo} }
This brief abstract introduces ROSLab, the block-based modular programming language
developed for PPR and referenced in the above paper.

\subsubsection*{On Embeddability of Modular Robot Designs \cite{mantzouratos2014embeddability}}
This is my paper (to appear in ICRA 2015 (hopefully)).  We formalize the notion of
design embeddability, and present an algorithm to efficiently detect it.  If a group of modules capable of a kinematic
task (for example, a planar arm) can be realized in a larger group of modules configured for a different kinematic
task, we say that the larger group \textit{embeds} the smaller.
If one design embeds another, it is guaranteed to replicate the kinematics of the
other design.  However, workspace constraints (self-collision) and dynamics/forces
are not considered.



\section*{Techniques and Algorithms}

\subsubsection*{FCL: A General Purpose Library for Collision and Proximity Detection
\cite{pan2012fcl}}
The Flexible Collision Library (FCL) is used by the ROS MoveIt package, and seems
to be one of the best freely available packages for collision detection. The paper
provides a nice overview of state-of-the-art collision detection techniques, many
of which are provided by FCL.
 
\bibliographystyle{plain}
\bibliography{../references}

\end{document}
