\title{Configuration and Controller Verification for SMORES modular robot}
\date{\today}

\documentclass[12pt]{article}

\begin{document}
\maketitle

\begin{abstract}
Key points:
\begin{itemize}
\item Build complex configurations and controllers from a library of basic configurations and controllers by arranging them in (nested) parallel and series structures
\item Both script and graphical user interface methods are implemented to build configurations and controllers
\item Verify that there is no self collision in composed configurations and during the execution  of controllers without simulation in Gazebo
\item Define ``unexpected behaviors'' due to instability of the configuration under a controller, and verify that there is no ``unexpected behaviors'' during the execution of controllers without simulation in Gazebo
\item Human asistant configuration and controller design (synthesis).
\item Show experiment in simulator. (Maybe with realy robot)
\end{itemize}
\end{abstract}

\section{Action Items}
\begin{itemize}
\item Check literatures about quasi stability - Tarik
\item Define unexpected behaviors - Jim
\item Get final draft at the beginning of Januray - All
\item Think about how meta-module is related with this paper - Tarik
\item Motion planner for modular robot - Jim
\item Come up with set of configurations and controllers - Shangyi
\end{itemize}

\section{Introduction}
\begin{itemize}
\item Introduce existing designs on modular robots.
\item Introduce existing works on modular robot controller design
\item Introduce SMORES modular robot and advantages
\item Introduce the contribution of this paper
\item Why fully automomous approach does not work well so far
\end{itemize}


\section{Preliminary}

\paragraph{SMORES robot module}
Define the ability of motion and connectivity of a SMORES robot module. Position and velocity of each Dof. Can be connected to four other modules at the same time. Representation of properties of a SMORES module, e.g. joint angels, global positions, connection information

\paragraph{Configuration}
Define the representation of a configuration as a set of SMORES robot modules connected in a certain way. Define topology graph.


\paragraph{Controller}
Define controller as a basic feedback controller for each Dof of each SMORES module. The reference input of the controller is a gait table. Define how the gait table is executed. Mention that controller in this paper refers to the input gait table.


\paragraph{Collision}
Define a collision between SMORES modules.

\paragraph{Controller conflict}
Define a conflict between controllers, i.e. giving opposite commands to the same Dof of a module at the same time.

\paragraph{Unexpected behavior}
Define the unexpected behavior of a configuration due to instability during a controller execution.

\section{Approach and Algorithm}
\paragraph{Configuration composition}
Define the composition of a set of configurations to a single configuration.

\subsection{Configuration Composition}
\paragraph{Input}
A set of configurations. A topology graph representing the connectivity among those configurations. A base module (for position transformation).
\paragraph{Output}
A composed configuration if it is safe.
\paragraph{Procedure}
\begin{itemize}
\item Start from the configurations that connects to the configuration with base module, transform their positions based on the position of the base configuration and topology graph.
\item Check if there is any collisions among the modules and report such collision.
\item \textbf{Check if the final configuration is stable. If not, find the plane that will make the configuration stable and transform the configuration.}
\item Show the expected behavior in simulator.
\end{itemize}

\paragraph{Controller composition}
Define the composition of a set of controllers to a single controller. Define the difference between a parallel composition and a series composition. Define the control composition graph.

\subsection{Controller Composition}
\paragraph{Input}
A configurations. A set of controllers. A control composition graph.
\paragraph{Output}
A composed controller if it is safe.
\paragraph{Procedure}
\begin{itemize}
\item Compose the set of controllers based on the given control composition graph. Explain how the parallel composition and series composition are handled.
\item\textbf{Check there is no controller conflict in the composition.}
\item Execute the composed controller in user defined incremental time interval. At each time step, update each module position and check collision.
\item \textbf{At each time step, check if the configuration will not have any unexpected behavior.}
\end{itemize}

\subsection{Complexity}
Discuss the complexity of the algorithm with respect to the number of modules and size of gait tables.

\section{Example and Experiment}
With simulation in Gazebo:
\begin{itemize}
\item Show a configuration composed from a set of basic configurations.
\item Show a composed controller that results in a collision in the configuration.
\item Show an updated controller that resolves the collision
\item Show a composed controller that results in an unexpected behavior.
\item Show an updated controller that eliminates the unexpected behavior.
\end{itemize}

\section{Conclusions}
We worked hard, and had fun.

\section{Future}
\begin{itemize}
\item How to represent different attribute/ability of the configurations
\end{itemize}

\end{document}
